\documentclass[%
    10pt,
    professionalfont,
    aspectratio=169,
    %handout,
]{beamer}

\mode<presentation>{%
    \usetheme[alternativetitlepage=bild]{Rub}
    \titlegraphic{img-bg-rub.jpg}
    \setbeamercovered{invisible}
}
\usepackage{pgfpages}
\setbeameroption{show notes on second screen}

\usepackage{rubfonts2009}
\usepackage[charter]{mathdesign}
\SetMathAlphabet{\mathcal}{normal}{OMS}{zplm}{m}{n}

\usepackage[utf8]{inputenc}
\usepackage[T1]{fontenc}

\usepackage{graphicx}
\definecolor{gelbgruen}{cmyk}{0.5,0,1,0}
\definecolor{lichtgrau}{cmyk}{0.03,0.03,0.03,0.1}
\definecolor{saphierblau}{cmyk}{1,0.5,0,.6}
\definecolor{ucorange}{cmyk}{0,0.45,0.93,0.04}
\definecolor{alertred}{rgb}{0.80,0.12,0.12}

\usepackage{abbrv}
\usepackage{delimiters}

\usepackage[english]{babel}
\usepackage{csquotes}

\usepackage{amsmath}
\usepackage{mathtools}

\usepackage{etoolbox,siunitx}
\sisetup{binary-units}

\usepackage{booktabs}

\usepackage{setspace}
\usepackage{todonotes}

\usepackage{algorithm}
\usepackage[noend]{algpseudocode}
\newcommand*\Let[2]{\State{} #1 $\gets$ #2}
\algrenewcommand\alglinenumber[1]{%
    {\sffamily\footnotesize#1}}
\algrenewcommand\algorithmicrequire{\textbf{Input:}}
\algrenewcommand\algorithmicensure{\textbf{Output:}}

\usepackage{accsupp}
\usepackage{listings}

\newcommand{\noncopynumber}[1]{%
    \BeginAccSupp{method=escape,ActualText={}}
    #1
    \EndAccSupp{}
}

\usepackage{tikz}
\usepackage{tikzsymbols}
\usetikzlibrary{backgrounds}
\usetikzlibrary{calc}
\usetikzlibrary{positioning}
\usetikzlibrary{decorations.pathreplacing, overlay-beamer-styles}
\usetikzlibrary{shapes}

\usetikzlibrary{chains,shapes.arrows,fit}
\definecolor{arrowcolor}{RGB}{201,216,232}% color for the arrow filling
\definecolor{circlecolor}{RGB}{79,129,189}% color for the inner circles filling
\colorlet{textcolor}{white}% color for the text inside the circles
\colorlet{bordercolor}{white}% color for the outer border of circles

\usetikzlibrary{crypto.symbols}
\tikzset{shadows=no}        % Option: add shadows to XOR, ADD, etc.
\usepackage{pgfplots}
\usetikzlibrary{pgfplots.colormaps}


\newcommand{\repeatarrow}{%
    \begin{tikzpicture}[inner sep=0pt, baseline=(base)]%
    \node (n) {};
    \draw[thick,<-] (n.center) ++(110:0.6em) arc (110:430:0.6em);
    \node (base) at (0,-.5ex) {};
    \end{tikzpicture}%
}

\usepackage[beamer]{hf-tikz}
\usepackage{forest}

\tikzset{%
    invisible/.style={opacity=0,text opacity=0},
    visible on/.style={alt={#1{}{invisible}}},
    alt/.code args={<#1>#2#3}{%
        \alt<#1>{\pgfkeysalso{#2}}{\pgfkeysalso{#3}} % \pgfkeysalso doesn't change the path
    },
    marked/.style={
        color=alertred,
    },
    marked on/.style={alt=#1{marked}{}},
}
\forestset{%
    visible on/.style={%
        for tree={%
            /tikz/visible on={#1},
            edge+={/tikz/visible on={#1}}
        }
    }
}

\def\colorize<#1>{%
    \temporal<#1>{\color{black}}{\color{alertred}}{\color{black!50}}%
}



\pgfdeclarelayer{background}
\pgfsetlayers{background,main}

\colorlet{arrowcolor}{saphierblau}
\colorlet{circlecolor}{white}
\colorlet{bordercolor}{gelbgruen}
\colorlet{textcolor}{saphierblau}


\setbeamertemplate{bibliography item}[text]
\usepackage[%
    backend=biber,
    bibencoding=ascii,
    style=alphabetic,
    sortcites,
    sorting=ynt,
    hyperref=true,
    maxbibnames=10,
    maxcitenames=5,
    giveninits=true,
]{biblatex}
\DefineBibliographyStrings{english}{%
    andothers = {\etal/}
}
\DeclareFieldFormat{eprint:iacr}{%
\mkbibacro{iacr}\addcolon\space{}
    \href{https://eprint.iacr.org/#1}{\nolinkurl{#1}}
}
\DeclareFieldFormat{eprint:iacrconf}{%
\mkbibacro{iacr}\addcolon\space{}
    \href{https://www.iacr.org/archive/#1}{\nolinkurl{#1}}
}
\renewcommand\bibfont{\scriptsize}
\bibliography{bibliography}

\hypersetup{%
    colorlinks=true,
    citecolor=black!70!green,
    linkcolor=black!70!red,
    urlcolor=black!20!blue,
}

\newcommand{\blfootnote}[1]{%
    \begingroup
        \renewcommand\thefootnote{}\footnote{#1}%
        \addtocounter{footnote}{-1}%
    \endgroup
}
\newcommand{\vpPp}{\vphantom{Pp}}
\newcommand{\coset}[2]{#1 \!\! + \!\! #2}
\newcommand{\through}[1]{\stackrel{#1}{\rightarrow}}
\newcommand\tower[2]{\genfrac{}{}{0pt}{}{#1}{#2}}
\DeclareMathOperator{\Prob}{Pr}

\newcommand{\F}{\mathbb{F}}
\newcommand{\derive}[2]{\Delta_{#1}\parens{#2}}
\DeclareMathOperator*{\diffOp}{\rightrightarrows}
\newcommand{\propDiff}[4]{#1 \diffOp^{#2}_{#3} #4}
\DeclareMathOperator{\SpanOp}{Span}
\newcommand{\Span}[1]{\SpanOp\set{#1}}

\makeatletter
\def\maxwidth#1{\ifdim\Gin@nat@width>#1 #1\else\Gin@nat@width\fi}
\def\maxheight#1{\ifdim\Gin@nat@height>#1 #1\else\Gin@nat@height\fi}
\makeatother

\def\printcipher/{PRINTcipher}
\def\robin/{Robin}
\def\iscream/{iSCREAM}
\def\scream/{SCREAM}
\def\midori/{Midori}
\def\misty/{Misty}
\def\zorro/{Zorro}
\def\spook/{\textsf{Spook}}
\def\clyde/{Clyde}
\def\shadow/{Shadow}

\title{Security Arguments and\\ Tool-based Design of Block Ciphers}
\subtitle{PhD Defense}
\author{Friedrich Wiemer}
\institute{%
    %SymCrypt, HGI, RUB
    Arbeitsgruppe Symmetrische Kryptographie, Horst-Görtz-Institut für IT Sicherheit, Ruhr-Universität Bochum
}

\date[December 13th, 2019]{\small{}December 13th, 2019}

\begin{document}

\begin{frame}[plain]
    \titlepage{}
\end{frame}

\section{Introduction}
\begin{frame}{The setting}{Block Ciphers and Security Notion}
\end{frame}

\begin{frame}{Block Ciphers}
\end{frame}

\begin{frame}{Security}
\end{frame}

\begin{frame}{Substitution Permutation Networks}
\end{frame}

\begin{frame}{Overview}
    \tableofcontents{}
\end{frame}

\section{Subspace Trail Attack}
\begin{frame}{Subspace Trail Cryptanalysis}
    \begin{block}{Main Idea of Subspace Trails}
        \centering
        \vspace{0.25em}
        \begin{tikzpicture}[scale=0.8]
            \tikzstyle{every node}=[transform shape];

            \node (left2-space) [draw,rectangle,thick,rounded corners,minimum width=3cm,minimum height=4cm,fill=white] at (1,0) {};
            \draw[thick] (left2-space.west)+(0,1.125cm) -- node[above, yshift=1.5mm] {$\coset{U}{a_r}$} +(3cm,1.125cm);
            \draw[thick] (left2-space.west)+(0,-0.25cm) -- node[above, yshift=5mm] {\dots} +(3cm,-0.25cm);
            \draw[thick] (left2-space.west)+(0,-1.125cm) -- node[above, yshift=1.5mm] {$\coset{U}{a_1}$}
                                                            node[below, yshift=-1.5mm] {$U$} +(3cm,-1.125cm);

            \node (middle-space) [draw,rectangle,thick,rounded corners,minimum width=3cm,minimum height=4cm,fill=white] at (5,0) {};
            \draw[thick] (middle-space.north)+(0pt,-0.5pt) -- node[above, yshift=0.5mm, rotate=45] (vbs) {$\coset{V}{b_s}$} +(-1.5cm,-1.5cm);
            \draw[thick] (middle-space.east)+(-0.5pt,1cm) -- node[above, yshift=10mm, rotate=45] (middle-dots) {\dots} +(-3cm+1.25pt,-2cm+1.25pt);
            \draw[thick] (middle-space.east)+(-0.5pt,-5mm) -- node[above, yshift=2.5mm, rotate=45] (vb1) {$\coset{V}{b_1}$}
                                                        node[below, yshift=-0.5mm, rotate=45] (v) {$V$} +(-1.5cm,-2cm);

            \draw[-latex] (1.75,-1.5) to [bend left] (4.75,-1.5);
            \visible<2->{%
            \draw[-latex] (1.75,-0.5) to [bend left] (6.325,-1.0);
            \draw[-latex] (1.75,+1.5) to [bend left] (middle-dots);
            }

            \node (left-f) at (3,0) {$F$};

            \visible<3->{%
            \node (right-space) [draw,rectangle,thick,rounded corners,minimum width=3cm,minimum height=4cm,fill=white] at (9,0) {};
            \draw[thick] (right-space.north)+(0pt,-0.5pt) -- node[above, yshift=0.5mm, rotate=-45] (wct) {$\coset{W}{c_t}$} +(+1.5cm,-1.5cm);
            \draw[thick] (right-space.west)+(0.5pt,1cm) -- node[above, yshift=10mm, rotate=-45] (right-dots) {\dots} +(3cm-1.25pt,-2cm+1.25pt);
            \draw[thick] (right-space.west)+(0.5pt,-5mm) -- node[above, yshift=2.5mm, rotate=-45] (wc1) {$\coset{W}{c_1}$}
                                                        node[below, yshift=-0.5mm, rotate=-45] (w) {$W$} +(+1.5cm,-2cm);

            \node (middle-f) at (7,0) {$F$};
            \node (right-f) at (11,0) {$\ldots$};

            \draw[-latex] (vb1) to [bend left] (w);
            \draw[-latex] (v) to [bend left] (wc1);
            \draw[-latex] (vbs) to [bend right] (9.625,+1.75);
            }
        \end{tikzpicture}
    \end{block}
    \visible<3->{%
    \begin{block}{Subspace Trail Cryptanalysis~[GRR16] (FSE'16)}
        \vspace{0.25em}
        \centering
        Let $U_0, \ldots, U_r \subseteq \F_2^n$, and $F : \F_2^n \to \F_2^n$.
        Then these form a \emph{subspace trail} (ST), $U_0 \through{F} \cdots \through{F} U_r$, iff
        \begin{equation*}
            \forall a \in U_i^\perp : \exists b \in U_{i+1}^\perp : \qquad F(\coset{U_i}{a}) \subseteq \coset{U_{i+1}}{b}
        \end{equation*}
    \end{block}
    }
\end{frame}

\begin{frame}{Our Goal}
\end{frame}

\begin{frame}{Subspace Propagation}
    \centering
    \begin{minipage}{0.45\textwidth}
    Given a starting subspace $U$, we can efficiently compute the corresponding longest subspace trail.

    \begin{lemma}
        \vspace{14pt}
        Let $U \through{F} V$ be a ST\@.
        Then for all $u \in U$ and all~$x$: $F(x) + F(x + u) \in V$.
        \vspace{14pt}
    \end{lemma}
    \end{minipage}
    \begin{minipage}{0.45\textwidth}
    \visible<2->{%
    \begin{block}{Proof}
        \centering
        \begin{tikzpicture}[scale=0.75]
            \tikzstyle{every node}=[transform shape];

            \node (left2-space) [draw,rectangle,thick,rounded corners,minimum width=3cm,minimum height=4cm,fill=white] at (1,0) {};
            \draw[thick] (left2-space.west)+(0,1.125cm) -- node[above, yshift=1.5mm] {$\coset{U}{a_s}$} +(3cm,1.125cm);
            \draw[thick] (left2-space.west)+(0,-0.25cm) -- node[above, yshift=5mm] {\dots} +(3cm,-0.25cm);
            \draw[thick] (left2-space.west)+(0,-1.125cm) -- node[below, yshift=-1.5mm] {$U$} +(3cm,-1.125cm);

            \node (right2-space) [draw,rectangle,thick,rounded corners,minimum width=3cm,minimum height=4cm,fill=white] at (5,0) {};
            \draw[thick] (right2-space.north)+(0pt,-0.5pt) -- node[above, yshift=0.5mm, rotate=45] {$\coset{V}{b_t}$} +(-1.5cm,-1.5cm);
            \draw[thick] (right2-space.east)+(-0.5pt,1cm) -- node[above, yshift=10mm, rotate=45] {\dots} +(-3cm+1.25pt,-2cm+1.25pt);
            \draw[thick] (right2-space.east)+(-0.5pt,-5mm) -- node[below, yshift=-0.5mm, rotate=45] {$V$} +(-1.5cm,-2cm);

            \node[xshift=-20pt, yshift=-18pt] (x1) at (left2-space) {$\cdot$};
            \node[above] (x1-label) at (x1) {$x$};

            \node[xshift=32pt, yshift=-4pt] (y1) at (right2-space) {$\cdot$};
            \node[above] (y1-label) at (y1) {$F(x)$};

            \draw[-latex] (x1) to [bend left] node[below,yshift=-5pt] {$F$} (y1);

            \visible<3->{%
                \node[xshift=25pt, yshift=-24pt] (x2) at (left2-space) {$\cdot$};
                \node[above] (x2-label) at (x2) {$x+u$};

                \node[xshift=-12pt, yshift=-42pt] (y2) at (right2-space) {$\cdot$};
                \node[below] (y2-label) at (y2) {$F(x+u)$};

                \draw[-latex] (x1) -- node[below,draw,fill=white,inner sep=1pt,circle] (u-label) {$u$} (x2);
                \node[xshift=17pt, yshift=-23pt] at (u-label) (u) {$\cdot$};

                \draw[-latex] (u-label) -- (u);

                \draw[-latex] (x2) to [bend left] (y2);
            }

            \visible<4->{%
                \draw[-latex] (y1) -- node[below,xshift=2pt,draw,fill=white,inner sep=1pt,circle] (v-label) {$v$} (y2);
                \node[xshift=22pt, yshift=-9pt] at (v-label) (v) {$\cdot$};

                \draw[-latex] (v-label) -- (v);
            }

        \end{tikzpicture}
    \end{block}
    }
    \end{minipage}
    \visible<5->{%
    \begin{minipage}{0.925\textwidth}
    \begin{block}{Computing the subspace trail}
        \begin{itemize}
            \item To compute the next subspace, we have to compute the image of the derivatives.
        \end{itemize}
    \end{block}
    \end{minipage}
    }
\end{frame}

\begin{frame}{Propagate a Basis}
\end{frame}

\begin{frame}{\texttt{ComputeTrail} Algorithm}
    \begin{minipage}{0.55\textwidth}
    \begin{block}{Computation of Subspace Trails}
    \begin{algorithmic}[1]
        \Require{A nonlinear function $F : \F_2^n \to \F_2^n$, a subspace $U$.}
        \Ensure{A subspace trail $\propDiff{U}{F}{}{\propDiff{\cdots}{F}{}{V}}$.}
        \Statex{}
        \Function{\texttt{ComputeTrail}}{$F$, $U$}
        \If{$\dim U = n$} \Return{$U$} \EndIf{}
        \State{}$V \leftarrow \emptyset$
        \For{$u_i$ basis vectors of $U$}
            \For{enough $x \in_\mathrm{R} \F_2^n$}\label{st:alg:compute_trail_1line}
                \Let{$V$}{$V \cup \derive{u_i}{F}(x)$}\label{st:alg:compute_trail_2line}
            \EndFor{}
        \EndFor{}
        \State{}$V \leftarrow \Span{V}$
        \State{}\Return{$\propDiff{U}{F}{}{\texttt{ComputeTrail}(F, V)}$}
        \EndFunction{}
    \end{algorithmic}
    \end{block}
    \end{minipage}%
    \hspace*{-10pt}%
    \begin{minipage}{0.5\textwidth}
        \begin{itemize}
            \item[] \textbf{Correctness}: previous two lemmata
            \item[] \textbf{Runtime}:
                \begin{itemize}
                    \item Line 4: max.\ $n$ iterations
                    \item Line 5: $n+c$ random vectors are enough
                    \item Overall: $\mathcal{O}(n^2)$ evaluations of $F$
                \end{itemize}
            \item[] \vspace{1em}Remaining Problem: cyclic STs
        \end{itemize}
    \end{minipage}
\end{frame}
\note{
    How many random vectors are enough:
    \url{https://math.stackexchange.com/questions/564603/probability-that-a-random-binary-matrix-will-have-full-column-rank}
}

\section{Security against Subspace Trail Attacks}
\begin{frame}{How to Bound the Length of Subspace Trails}
\end{frame}

\begin{frame}{Activating a single S-box only}
\end{frame}

\begin{frame}{The Connection to Linear Structures}
\end{frame}

\begin{frame}{S-boxes without Linear Structures}
\end{frame}

\begin{frame}{S-boxes with Linear Structures}
\end{frame}

\section{Conclusion}
\begin{frame}{Conclusion}{Thanks for your attention!}
    \centering
    \begin{minipage}{0.5\textwidth}
    \begin{block}{Applications of \textsc{ComputeTrail}}
        \begin{itemize}
            \item Bound longest probability-one subspace trail
            \item Link to Truncated Differentials
            \item Finding key-recovery strategies
        \end{itemize}
    \end{block}
    \end{minipage}
    \begin{minipage}{0.45\textwidth}
        \centering
        \begin{figure}[!htb]
            \includegraphics[height=50mm]{data/flickr/questionmark.png}
        \end{figure}
    \end{minipage}
\end{frame}

\begin{frame}[allowframebreaks]{References}
    \tiny
    \printbibliography{}
\end{frame}

\end{document}
