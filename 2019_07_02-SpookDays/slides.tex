\section{Round Constants}
\begin{frame}{Invariant Attacks}
    What are Invariant Attacks
\end{frame}

\begin{frame}{Invariant Attacks}{Proving Resistance}
    \begin{itemize}
        \item Goal: apply security argument from
              \begin{quote}
                  \fullcite{C:BCLR17}.
              \end{quote}
        \item This argument proves that there is no invariant for both, the S-box and linear layer, parts of the round function.
        \item However, there might be other partitionings of the round function, for which there are invariants (in particular, Christof Beierle found some examples).
        \item It is not clear how to prove the general absence of invariant attacks; this is the best we can currently prove.
        \item All known attacks exploit exactly this structure (that is, splitting in S-box and linear layer).
    \end{itemize}
\end{frame}

\begin{frame}{Invariant Attacks}{Recap Security Argument}
\only<1>{%
    \begin{itemize}
        \item The argument bases on the observation that all published invariant attacks use invariant functions which are invariant for the S-box layer \emph{and} invariant for the linear layer.
        \item Furthermore, invariants over the linear layer $L$ and the round key addition have to be invariant over $W_L(c_1, \dots, c_t)$.
        \item $W_L(c_1, \dots, c_t)$ is the smallest $L$-invariant subspace of $\F_2^n$ containing all $c_i$, where these are the round constant differences from rounds that add the same round key.
        \item $W_L(c_1, \dots, c_t)$ is $L$-invariant if and only if: $\forall x \in W_L(c_1, \dots, c_t): L(x) \in W_L(c_1, \dots, c_t)$.
        \item Thus, if $W_L(c_1, \dots, c_t)$ contains the whole $\F_2^n$, only the trivial invariants for $L$ and the key addition remain (the constant 0 and 1 functions).
        \item There is a link between the invariant factors $Q_i$ of the linear layer and the dimension of~$W_L$,~\cite[Theorem~1]{C:BCLR17}:
              \begin{equation*}
                  \max_{c_1, \dots, c_t \in \F_2^n} \dim W_L(c_1, \dots, c_t) = \sum_{i=1}^t \deg Q_i\;.
              \end{equation*}
    \end{itemize}
}

\only<2>{%
    For \clyde/:
    \begin{itemize}
        \item The linear layer has four invariant factors ($4 \times (x^{32}+1)$).
        \item Due to its tweakey schedule, every tweakey equals the fourth next tweakey: $\mathrm{TK}_i = \mathrm{TK}_{i+3}$.
        \item After each step (two rounds), a tweakey is added.
        \item We need at least four round constant differences; looking at the round constant additions, this implies at least three steps (six rounds), so that $W_L$ can achieve full dimension.
        \item In particular, the set of round constant differences, for the six steps \clyde/ uses, is:
              \begin{align*}
                  D &= D_{\mathrm{TK}_0} \cup D_{\mathrm{TK}_1} \cup D_{\mathrm{TK}_2} \cup D_0\\[5pt]
                  D_{\mathrm{TK}_0} &= \set{0 + W(5), 0 + W(11), W(5) + W(11)} \\
                  D_{\mathrm{TK}_1} &= \set{W(1) + W(7)} \\
                  D_{\mathrm{TK}_2} &= \set{W(3) + W(9)} \\
                  D_{0} &= \set{a + b \given a, b \in \set{W(0), W(2), W(4), W(6), W(8), W(10)}, a \neq b}
              \end{align*}
        \item This gives us 20 round constant differences.
    \end{itemize}
}

\only<3>{%
    For \clyde/ (cont.):
    \begin{itemize}
        \item Computing $W_L$ is efficiently doable (takes $\approx$ 10 seconds on my laptop).
        \item For the round constants chosen for \clyde/, $\dim W_L(D) = 128 = n$.
        \item Thus, we can apply:
              \begin{block}{Proposition~2 (\cite{C:BCLR17})}
                  Suppose that the dimension of $W_L(D)$ is at least $n-1$.
                  Then any invariant $g$ is linear or constant.
                  As a consequence, there is no non-trivial invariant $g$ of the S-box layer, unless the S-box layer has a component of degree 1.
              \end{block}
        \item Such an S-box would be attackable by linear cryptanalysis.
        \item We conclude that we cannot find any $g$ for \clyde/ which is at the same time invariant for the S-box layer and for the linear layer.
    \end{itemize}
}
\end{frame}

\section{Subspace Trails}
\begin{frame}{Subspace Trails}
    What are Subspace Trails
\end{frame}

\begin{frame}{Subspace Trails}{Proving Resistance}
    \begin{itemize}
        \item Goal: apply security argument from
              \begin{quote}
                  \fullcite{ToSC:LeaTezWie18}.
              \end{quote}
    \end{itemize}
\end{frame}

\begin{frame}{Subspace Trails}{Recap Security Argument}
    \begin{itemize}
        \item Basically: Exhaustive Search of possible subspace trails
        \item Reduce tested subspace trails to a minimal set, so that all subspace trails are still covered
        \item For SPN constructions using S-boxes with linear structures, this is the set
              \begin{equation*}
                  \mathcal{W} \coloneqq \set{W_{i,\alpha} \coloneqq \set{0}^{i-1} \times \set{0,\alpha} \times \set{0}^{k-i} \given \alpha \in \F_2^n, 1 \leq i \leq k}.
              \end{equation*}
              where the round function applies $k$ S-boxes in parallel and each S-box permutes $\F_2^n$
        \item That is, we check for each candidate starting subspace $\set{W_{i,\alpha}}$, the length of the corresponding subspace trail, using the \textsc{Generic Subspace Trail Length} algorithm from \textcite{ToSC:LeaTezWie18}.
        \item Intuitively, the $W_{i,\alpha}$ capture all possible output values after the first S-box layer, when only one S-box is active, the algorithm then checks the longest possible subspace trail length from this point on.
    \end{itemize}
\end{frame}

\begin{frame}{Subspace Trails}{Recap Security Argument -- The algorithms}
\centering
\only<1>{%
    \begin{block}{Notation}
    \begin{itemize}
        \item $\Delta_\alpha(F) \coloneqq x \mapsto F(x) + F(x+\alpha)$, the derivative of $F$ in direction $\alpha$
        \item $F^k \coloneqq x \mapsto (\underbrace{F(x), \ldots, F(x)}_{\text{$k$ times}})$, the $k$-th parallel application of $F$ (\eg/ an S-box layer)
    \end{itemize}
    \end{block}
}

\only<2>{%
    \begin{minipage}{0.7\textwidth}
    \centering
    \begin{block}{Compute subspace trails}
    \begin{algorithmic}[1]
        \Require{A nonlinear, bijective function $F : \F_2^n \to \F_2^n$ and a subspace $U$.}
        \Ensure{The longest subspace trail starting in $U$ over $F$.}
        \Statex{}
        \Function{Compute Trail}{$F$, $U$}
        \If{$\dim(U) = n$}
            \State{}\Return{$U$}
        \EndIf{}
        \State{}$V \leftarrow \emptyset$
        \For{$u_i$ basis vectors of $U$}
            \For{enough $x \in_\mathrm{R} \F_2^n$}\label{alg:compute_trail_1line}
                \State{} $V \leftarrow V \cup \Delta_{u_i}(F)(x)$\label{alg:compute_trail_2line}
            \EndFor{}
        \EndFor{}
        \State{}$V \leftarrow \mathrm{span}(V)$
        \State{}\Return{the subspace trail $U \rightarrow \textsc{Compute Trail}(F, V)$}
        \EndFunction{}
    \end{algorithmic}
    \end{block}
    \end{minipage}
}

\only<3>{%
    \begin{minipage}{0.7\textwidth}
    \centering
    \begin{block}{Generic Subspace Trail Search}
    \begin{algorithmic}[1]
        \Require{A linear layer matrix $M : \F_2^{n \cdot k \times n \cdot k}$, and an S-box $S : \F_2^n \to \F_2^n$.}
        \Ensure{A bound on the length of all subspace trails over $F = M \circ S^k$.}
        \Statex{}
        \Function{Generic Subspace Trail Length}{$M$, $S$}
        \State{}empty list $L$
        \For{possible initial subspaces represented by $W_{i,\alpha} \in \mathcal{W}$}
            \State{}$L.\mathrm{append}(\textsc{Compute Trail}(S^k \circ M, \set{W_{i,\alpha}}))$
        \EndFor{}
        \State{}\Return{$\max{\set{\mathrm{len}(t) \given t \in L}}$}
        \EndFunction{}
    \end{algorithmic}
    \end{block}
    \end{minipage}
}
\end{frame}

%\subsection{Result}
%
%\begin{table}
%    \centering
%    \caption{%
%        Bound on the longest subspace trails in \clyde/ and \shadow/ (actual longest subspace trails can be one round longer).
%    }\label{tab:results_st}
%    \renewcommand{\arraystretch}{1.2}
%    \begin{tabular}{lcccc}
%        \toprule
%                 & \multicolumn{4}{c}{\cref{alg:generic}} \\
%        Cipher   & $r_e$   &   $d$   &    $r_d$   &  $d$  \\
%        \midrule
%        \clyde/  &   2     &    44   &      2     &   41  \\ \rowcolor{gray!10}
%        \shadow/ &  ???    &   ???   &     ---    &  ---  \\
%        \bottomrule
%    \end{tabular}
%\end{table}
%
%See \cref{tab:results_st}.

\section{Division Property}
\begin{frame}{Division Property}
    \begin{itemize}
    \item Goal: apply security argument fro
        \begin{quote}
            \fullcite{AC:XZBL16}.
        \end{quote}
    \item Approach: model division trail propagations as MILP, find solutions for this over increasing number of rounds.
    \end{itemize}
\end{frame}

\section{Results}
\begin{frame}{Results}
    \centering
    \begin{minipage}{0.5\textwidth}
    \begin{block}{Number of rounds for which a distinguisher exist}
    \centering
    \renewcommand{\arraystretch}{1.2}
    \begin{tabular}{lcc}
        \toprule
        Cipher   & Subspace Trails & Division Property \\
        \midrule
        \clyde/  &      2 (+1)     &         8         \\ \rowcolor{gray!10}
        \shadow/ &      4 (+1)     &        ???        \\
        \bottomrule
    \end{tabular}
    \end{block}
    \end{minipage}
\end{frame}
