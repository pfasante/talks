\section{Division Property}
\begin{frame}
    \centering
    \Huge
    Division Property
    \vfill
\end{frame}

\begin{frame}{Division Property}
    \begin{block}{Main Idea: Division Property}
        \begin{itemize}
            \item Generalisation of Integral and Higher Order Differential attacks
            \item Captures properties of bits in a set
            \item For standard integral attacks: zero-sum, all or constant
            \item The Division Property allows to capture properties \enquote{in between} these\\
                  (even if they do not have such a nice description as \eg/ the zero-sum)
        \end{itemize}
    \end{block}
    \begin{block}{Division Property}
        ???
    \end{block}
\end{frame}

\begin{frame}{Division Property}{Related Work}
    \begin{timeline}
        \Task[\cite{EC:Todo15}]{Publication of DP attack}
        \Task[\cite{C:Todo15}]{DP attack breaking\\ full \misty/}
        \Task[\cite{C:BouCan16}]{Analysis of DP, S-box properties to resist\\ this attack}
        \Task[\cite{FSE:TodMor16}]{Bit-based DP}
        \Task[\cite{AC:XZBL16}]{MILP based search of\\ DP distinguishers}
        \Task[%
                \cite{C:TIHM17}\\
                \cite{C:WHTLIM18}
            ]{DP based Cube attacks on stream ciphers}
    \end{timeline}
\end{frame}

\begin{frame}{Division Trails}
    \begin{block}{Division Trail}
        ???
    \end{block}
    \begin{block}{Propagating Bit-Based Division Trails}
    \begin{columns}
    \begin{column}{0.3\textwidth}
        \vspace*{-10pt}
        \begin{gather*}
            \mathrm{copy} : x \mapsto (x, x) \\
            \mathcal{D}^1_x \stackrel{\mathrm{copy}}{\to} \begin{cases*}
                \mathcal{D}^1_{(0,0)}       & if $x = 0$ \\
                \mathcal{D}^1_{(0,1),(1,0)} & if $x = 1$
            \end{cases*}
        \end{gather*}
        \vspace*{-3pt}
    \end{column}
    \begin{column}{0.3\textwidth}
        \vspace*{-10pt}
        \begin{gather*}
            \mathrm{xor} : (x,y) \mapsto x + y \\
            \mathcal{D}^{1,2}_{(k_0,k_1)} \stackrel{\mathrm{xor}}{\to} \mathcal{D}^1_{k_0+k_1}
        \end{gather*}
    \end{column}
    \begin{column}{0.3\textwidth}
        \centering
        S-box $S : \F_2^n \to \F_2^n$:\\
        see~\cite[Algorithm~2]{AC:XZBL16},
        computes for all $u \in \F_2^n$\\
        \vspace*{-15pt}
        \begin{equation*}
            \mathcal{D}^{1,n}_{u} \stackrel{S}{\to} \mathcal{D}^{1,n}_V
        \end{equation*}
        \st/ $u \to v$ is a DT $\forall v \in V$.
    \end{column}
    \end{columns}
    \end{block}
\end{frame}

\begin{frame}{Division Property}
    \begin{block}{\textbf{Goal}: Apply security argument from}
    \begin{quote}
        \fullcite{AC:XZBL16}.
    \end{quote}
    \end{block}
    \begin{exampleblock}{What do we get from this?}
        \centering
        Number of rounds for which a division property/integral distinguisher exists.
    \end{exampleblock}
    \begin{block}{Approach (similiar to Subspace Trails)}
        \begin{itemize}
            \item Pick starting DPs in a way that covers all possibilities
            \item Model division trail propagations as MILP
            \item Find solutions for this over increasing number of rounds
        \end{itemize}
    \end{block}
\end{frame}

\begin{frame}{Division Property}{MILP model}
    \begin{block}{MILP}
        What is a MILP
    \end{block}
    \begin{itemize}
        \item Objective function
        \item Starting DP
        \item Propagation Rules
        \item Stopping Rule
    \end{itemize}
\end{frame}
