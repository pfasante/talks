\documentclass[%
    10pt,
    professionalfont,
    aspectratio=169,
    %handout,
]{beamer}

\mode<presentation>{%
    \usetheme[alternativetitlepage=bild]{Rub}
    \titlegraphic{img-bg-rub.jpg}
    \setbeamercovered{invisible}
}
\usepackage{pgfpages}
%\setbeameroption{show notes on second screen}

\usepackage{rubfonts2009}
\usepackage[charter]{mathdesign}
\SetMathAlphabet{\mathcal}{normal}{OMS}{zplm}{m}{n}

\usepackage[utf8]{inputenc}
\usepackage[T1]{fontenc}

\usepackage{graphicx}
\definecolor{gelbgruen}{cmyk}{0.5,0,1,0}
\definecolor{lichtgrau}{cmyk}{0.03,0.03,0.03,0.1}
\definecolor{saphierblau}{cmyk}{1,0.5,0,.6}
\definecolor{ucorange}{cmyk}{0,0.45,0.93,0.04}
\definecolor{alertred}{rgb}{0.80,0.12,0.12}

\usepackage{abbrv}
\usepackage{delimiters}

\usepackage[english]{babel}
\usepackage{csquotes}

\usepackage{amsmath}
\usepackage{mathtools}

\usepackage{etoolbox,siunitx}
\sisetup{binary-units}

\usepackage{booktabs}

\usepackage{setspace}
\usepackage{todonotes}

\usepackage{algorithm}
\usepackage[noend]{algpseudocode}
\newcommand*\Let[2]{\State{} #1 $\gets$ #2}
\algrenewcommand\alglinenumber[1]{%
    {\sf\footnotesize#1}}
\algrenewcommand\algorithmicrequire{\textbf{Input:}}
\algrenewcommand\algorithmicensure{\textbf{Output:}}

\usepackage{accsupp}
\usepackage{listings}

\newcommand{\noncopynumber}[1]{%
    \BeginAccSupp{method=escape,ActualText={}}
    #1
    \EndAccSupp{}
}

\usepackage{tikz}
\usepackage{tikzsymbols}
\usetikzlibrary{backgrounds}
\usetikzlibrary{calc}
\usetikzlibrary{positioning}
\usetikzlibrary{decorations.pathreplacing, overlay-beamer-styles}
\usetikzlibrary{shapes}

\usetikzlibrary{chains,shapes.arrows,fit}
\definecolor{arrowcolor}{RGB}{201,216,232}% color for the arrow filling
\definecolor{circlecolor}{RGB}{79,129,189}% color for the inner circles filling
\colorlet{textcolor}{white}% color for the text inside the circles
\colorlet{bordercolor}{white}% color for the outer border of circles

\usetikzlibrary{crypto.symbols}
\tikzset{shadows=no}        % Option: add shadows to XOR, ADD, etc.
\usepackage{pgfplots}
\usetikzlibrary{pgfplots.colormaps}


\newcommand{\repeatarrow}{%
    \begin{tikzpicture}[inner sep=0pt, baseline=(base)]%
    \node (n) {};
    \draw[thick,<-] (n.center) ++(110:0.6em) arc (110:430:0.6em);
    \node (base) at (0,-.5ex) {};
    \end{tikzpicture}%
}

\usepackage[beamer]{hf-tikz}
\usepackage{forest}

\tikzset{%
    invisible/.style={opacity=0,text opacity=0},
    visible on/.style={alt={#1{}{invisible}}},
    alt/.code args={<#1>#2#3}{%
        \alt<#1>{\pgfkeysalso{#2}}{\pgfkeysalso{#3}} % \pgfkeysalso doesn't change the path
    },
    marked/.style={
        color=alertred,
    },
    marked on/.style={alt=#1{marked}{}},
}
\forestset{%
    visible on/.style={%
        for tree={%
            /tikz/visible on={#1},
            edge+={/tikz/visible on={#1}}
        }
    }
}



\newcounter{task}

\newlength\taskwidth% width of the box for the task description
\newlength\taskvsep% vertical distance between the task description and arrow

\setlength\taskwidth{2.75cm}
\setlength\taskvsep{20pt}

\def\taskpos{}
\def\taskanchor{}

\newcommand\task[1]{%
  {\parbox[t]{\taskwidth}{\scriptsize\centering#1}}}

\tikzset{
inner/.style={
  on chain,
  circle,
  inner sep=4pt,
  fill=circlecolor,
  line width=1.5pt,
  draw=bordercolor,
  text width=3.5em,
  align=center,
  text height=1.25ex,
  text depth=0ex
},
on grid
}

\newcommand\Task[2][]{%
\node[inner xsep=0pt] (c1) {\phantom{A}};
\stepcounter{task}
\ifodd\thetask\relax
  \renewcommand\taskpos{\taskvsep}\renewcommand\taskanchor{south}
\else
  \renewcommand\taskpos{-\taskvsep}\renewcommand\taskanchor{north}
\fi
\node[inner,font=\footnotesize\sffamily\color{textcolor}]    
  (c\the\numexpr\value{task}+1\relax) {#1};
\node[anchor=\taskanchor,yshift=\taskpos] 
  at (c\the\numexpr\value{task}+1\relax) {\task{#2}};
}

\newcommand\drawarrow{% the arrow is placed in the background layer 
                                                     % after the node for the tasks have been placed
\ifnum\thetask=0\relax
  \node[on chain] (c1) {}; % if no \task command is used, the arrow will be drawn
\fi
\node[on chain] (f) {};
\begin{pgfonlayer}{background}
\node[
  inner sep=10pt,
  single arrow,
  single arrow head extend=0.8cm,
  draw=none,
  fill=arrowcolor,
  fit= (c1) (f)
] (arrow) {};
\fill[white] % the decoration at the tail of the arrow
  (arrow.before tail) -- (c1|-arrow.west) -- (arrow.after tail) -- cycle;
\end{pgfonlayer}
}

\newenvironment{timeline}[1][node distance=.75\taskwidth]
  {\par\noindent\begin{tikzpicture}[start chain,#1]}
  {\drawarrow\end{tikzpicture}\par}

\pgfdeclarelayer{background}
\pgfsetlayers{background,main}

\colorlet{arrowcolor}{saphierblau}
\colorlet{circlecolor}{white}
\colorlet{bordercolor}{gelbgruen}
\colorlet{textcolor}{saphierblau}



\setbeamertemplate{bibliography item}[text]
\usepackage[%
    backend=biber,
    bibencoding=ascii,
    style=alphabetic,
    sortcites,
    sorting=ynt,
    maxbibnames=10,
    maxcitenames=5,
    giveninits=true,
]{biblatex}
\DefineBibliographyStrings{english}{%
    andothers = {\etal/}
}
\DeclareFieldFormat{eprint:iacr}{%
\mkbibacro{iacr}\addcolon\space{}
    \href{https://eprint.iacr.org/#1}{\nolinkurl{#1}}
}
\DeclareFieldFormat{eprint:iacrconf}{%
\mkbibacro{iacr}\addcolon\space{}
    \href{https://www.iacr.org/archive/#1}{\nolinkurl{#1}}
}
\renewcommand\bibfont{\scriptsize}
\bibliography{bibliography}

\newcommand{\blfootnote}[1]{%
    \begingroup
        \renewcommand\thefootnote{}\footnote{#1}%
        \addtocounter{footnote}{-1}%
    \endgroup
}
\newcommand{\vpPp}{\vphantom{Pp}}
\newcommand{\coset}[2]{#1 \!\! + \!\! #2}
\newcommand{\through}[1]{\stackrel{#1}{\rightarrow}}
\newcommand{\F}{\mathbb{F}}
\newcommand\tower[2]{\genfrac{}{}{0pt}{}{#1}{#2}}
\DeclareMathOperator{\Span}{span}
\DeclareMathOperator{\Prob}{Pr}

\makeatletter
\def\maxwidth#1{\ifdim\Gin@nat@width>#1 #1\else\Gin@nat@width\fi}
\def\maxheight#1{\ifdim\Gin@nat@height>#1 #1\else\Gin@nat@height\fi}
\makeatother

\def\printcipher/{PRINTcipher}
\def\robin/{Robin}
\def\iscream/{iSCREAM}
\def\scream/{SCREAM}
\def\midori/{Midori}
\def\zorro/{Zorro}
\def\spook/{\textsf{Spook}}
\def\clyde/{Clyde}
\def\shadow/{Shadow}

\title{Cryptanalysis of \clyde/ and \shadow/}
\subtitle{}
\author[Friedrich~Wiemer]{Gregor~Leander, and \emph{Friedrich~Wiemer}}
\institute{%
    Horst Görtz Institut für IT Sicherheit, Ruhr-Universität Bochum
}

\date{July 3rd, 2019}

\begin{document}

\begin{frame}
    \titlepage{}
\end{frame}

\begin{frame}{Overview}
    \tableofcontents{}
\end{frame}

\section{Invariant Attacks -- Round Constants}
\begin{frame}
    \centering
    \Huge
    Invariant Attacks
    \vfill
\end{frame}

\begin{frame}{Invariant Attacks}
    \begin{block}{Main Idea: Invariant Subspaces}
        \centering
        \vspace{0.25em}
        \begin{tikzpicture}[scale=0.8]
            \tikzstyle{every node}=[transform shape];

            \node (left-space) [draw,rectangle,thick,rounded corners,minimum width=3cm,minimum height=4cm,fill=white] at (1,0) {};
            \draw[thick] (left-space.west)+(0,1.125cm) -- node[above, yshift=1.5mm] (uar1) {$\coset{U}{a_r}$} +(3cm,1.125cm);
            \draw[thick] (left-space.west)+(0,-0.25cm) -- node[above, yshift=5mm] {\dots} +(3cm,-0.25cm);
            \draw[thick] (left-space.west)+(0,-1.125cm) -- node[above, yshift=1.5mm] {$\coset{U}{a_1}$}
                                                           node[below, yshift=-1.5mm] {$U$} +(3cm,-1.125cm);

            \node (middle-space) [draw,rectangle,thick,rounded corners,minimum width=3cm,minimum height=4cm,fill=white] at (5,0) {};
            \draw[thick] (middle-space.west)+(0,1.125cm) -- node[above, yshift=1.5mm] (uar2) {$\coset{U}{a_r}$} +(3cm,1.125cm);
            \draw[thick] (middle-space.west)+(0,-0.25cm) -- node[above, yshift=5mm] {\dots} +(3cm,-0.25cm);
            \draw[thick] (middle-space.west)+(0,-1.125cm) -- node[above, yshift=1.5mm] (ua12) {$\coset{U}{a_1}$}
                                                           node[below, yshift=-1.5mm] {$U$} +(3cm,-1.125cm);

            \draw[-latex] (uar1.east) -- node[above] {$F$} (ua12.west);
            \draw[-latex] (ua12.north) -- node[right] {$\oplus k_1$} (uar2.south);

            \visible<2->{%
            \node (right-space) [draw,rectangle,thick,rounded corners,minimum width=3cm,minimum height=4cm,fill=white] at (9,0) {};
            \draw[thick] (right-space.west)+(0,1.125cm) -- node[above, yshift=1.5mm] (uar3) {$\coset{U}{a_r}$} +(3cm,1.125cm);
            \draw[thick] (right-space.west)+(0,-0.25cm) -- node[above, yshift=5mm] {\dots} +(3cm,-0.25cm);
            \draw[thick] (right-space.west)+(0,-1.125cm) -- node[above, yshift=1.5mm] (ua13) {$\coset{U}{a_1}$}
                                                           node[below, yshift=-1.5mm] {$U$} +(3cm,-1.125cm);

            \draw[-latex] (uar2.east) -- node[above] {$F$} (ua13.west);
            \draw[-latex] (ua13.north) -- node[right] {$\oplus k_2$} (uar3.south);
            }
        \end{tikzpicture}
    \end{block}
    \visible<3->{%
    \begin{block}{Invariant Subspace Attacks~\cite{C:LAAZ11} (CRYPTO'11)}
        \vspace{0.25em}
        Let $U \subseteq \F_2^n$, $c, d \in U^\perp$, and $F : \F_2^n \to \F_2^n$.
        Then $U$ is an \emph{invariant subspace} (IS) if and only if
            $F(\coset{U}{c}) = \coset{U}{d}$
        and all round keys in $\coset{U}{(c+d)}$ are \emph{weak keys}.
    \end{block}
    }
\end{frame}

\begin{frame}{Invariant Attacks}{A Short History}
\begin{timeline}
\Task[\cite{C:LAAZ11}]{Publication of IS attack, breaking \printcipher/}
\Task[\cite{EC:LeaMinRon15}]{Generic Algorithm to find ISes, breaking \robin/, \iscream/, \zorro/}
\Task[\cite{ToSC:GJNQSM16}]{IS attack breaking \midori/64}
\Task[\cite{AC:TodLeaSas16}]{Invariant Set generalisation, breaking \scream/, \iscream/, \midori/64}
\Task[\cite{C:BCLR17}]{Proving resistance for Invariant attacks}
\end{timeline}
\end{frame}

\begin{frame}{Invariant Attacks}{Proving Resistance}
    \centering
    \begin{block}{\textbf{Goal}: Apply security argument from}
    \begin{quote}
        \fullcite{C:BCLR17}.
    \end{quote}
    \end{block}
    \begin{exampleblock}{What do we get from this?}
        \begin{itemize}
            \item Non-existence of invariants for both parts of the ronud function (S-box and linear layer)
        \end{itemize}
    \end{exampleblock}
    \begin{alertblock}{Issues}
    \begin{itemize}
        \item Other partitionings of the round function might allow invariants (Christof B\@. found examples)
        \item Not clear how to prove the general absence of invariant attacks (best we can currently prove)
        \item All known attacks exploit exactly this structure (splitting in S-box and linear layer)
    \end{itemize}
    \end{alertblock}
\end{frame}

\begin{frame}{Invariant Attacks}{Recap Security Argument (I)}
    \begin{columns}
        \begin{column}{0.35\textwidth}
        \begin{block}{Observation}
            \begin{itemize}
                \item Invariants for the linear layer $L$ and round key addition have to contain special linear structures.
                \item Denote by $c_1, \dots, c_t$ the round constant differences for rounds with the same round key.
                \item Then the linear structures of any invariant have to contain $W_L(c_1, \dots, c_t)$.
            \end{itemize}
        \end{block}
        \pause
    \end{column}
    \begin{column}{0.6\textwidth}
    \begin{block}{Linear Structures}
        Let $f : \F_2^n \to \F_2$.
        Then its \emph{linear structures} are
        \vspace*{-5pt}
        \begin{equation*}
            \mathrm{LS} \coloneqq \set{a \given f(x) + f(x+a) \text{ is constant}}\;.
        \end{equation*}
        \vspace*{-15pt}
    \end{block}
    \begin{block}{The smallest $L$-invariant subspace}
        $W_L(c_1, \dots, c_t)$ is the \emph{smallest $L$-invariant subspace} of~$\F_2^n$ containing all~$c_i$
        \vspace*{-5pt}
        \begin{equation*}
            \Leftrightarrow \forall x \in W_L(c_1, \dots, c_t): L(x) \in W_L(c_1, \dots, c_t)
        \end{equation*}
        \vspace*{-15pt}
    \end{block}
    \pause
    \begin{exampleblock}{The simple case}
        If $W_L(c_1, \dots, c_t) = \F_2^n$, only trivial invariants for $L$ and key addition are possible (constant 0 and 1 function).
    \end{exampleblock}
    \end{column}
    \end{columns}
\end{frame}

\begin{frame}{Invariant Attacks}{Recap Security Argument (II)}
    \begin{block}{Application to \clyde/}
    \begin{columns}
        \begin{column}{0.55\textwidth}
        \begin{itemize}
            \item Find the important round constant differences:\\
                  {\small (the differences where the same tweakey is added)}
        \end{itemize}
        \end{column}
        \pause
            \begin{column}{0.45\textwidth}
            \begin{itemize}
                \item[] Set of RC differences $D$ below\\
                        with $\abs{D} = 20$
            \end{itemize}
            \end{column}
    \end{columns}
    \end{block}
    \begin{columns}
        \visible<3->{
        \begin{column}{0.49\textwidth}
            \begin{tikzpicture}[background rectangle/.style={fill=white}, show background rectangle]
                \begin{scope}[scale=0.75, transform shape]
                \tikzstyle{every node}=[transform shape];
                \tikzstyle{every node}=[node distance=25pt];

                \node (XOR11) [XOR,scale=1.2] {};

                \node [left of=XOR11] (x) {$x$};
                \path[line] (x) edge (XOR11);

                \foreach \i in {1, 2, 3, 4} {
                    \node (f1\i) [right of=XOR1\i,draw,rectangle,thick,rounded corners,minimum width=0.25cm,minimum height=0.75cm] {$R$};
                    \path[line] (XOR1\i) edge (f1\i);
                    \pgfmathtruncatemacro\tmp{\i+1}
                    \node (XOR1\tmp) [right of=f1\i,XOR,scale=1.2] {};
                    \path[line] (f1\i) edge (XOR1\tmp);
                }

                \node (f21) [below=50pt of XOR15,draw,rectangle,thick,rounded corners,minimum width=0.25cm,minimum height=0.75cm] {$R$};
                \path[line] (XOR15.east) -- +(20pt,0) |- (f21.east);
                \node (XOR21) [left of=f21,XOR,scale=1.2] {};
                \path[line] (f21) edge (XOR21);

                \foreach \i in {2, 3, 4} {
                    \pgfmathtruncatemacro\tmp{\i-1}
                    \node (f2\i) [left of=XOR2\tmp,draw,rectangle,thick,rounded corners,minimum width=0.25cm,minimum height=0.75cm] {$R$};
                    \path[line] (XOR2\tmp) edge (f2\i);
                    \node (XOR2\i) [left of=f2\i,XOR,scale=1.2] {};
                    \path[line] (f2\i) edge (XOR2\i);
                }

                \node (f31) [below=50pt of XOR24,draw,rectangle,thick,rounded corners,minimum width=0.25cm,minimum height=0.75cm] {$R$};
                \path[line] (XOR24.west) -- +(-20pt,0) |- (f31.west);
                \node (XOR31) [right of=f31,XOR,scale=1.2] {};
                \path[line] (f31) edge (XOR31);

                \foreach \i in {2, 3, 4} {
                    \pgfmathtruncatemacro\tmp{\i-1}
                    \node (f3\i) [right of=XOR3\tmp,draw,rectangle,thick,rounded corners,minimum width=0.25cm,minimum height=0.75cm] {$R$};
                    \path[line] (XOR3\tmp) edge (f3\i);
                    \node (XOR3\i) [right of=f3\i,XOR,scale=1.2] {};
                    \path[line] (f3\i) edge (XOR3\i);
                }

                \node [right = 20pt of XOR34.east] (y) {$y$};
                \path[line] (XOR34) edge (y);

                \node (k00) [above=25pt of XOR11,marked on=<4>] {\small $\mathrm{TK}(0)$};
                \path[line,marked on=<4>] (k00) edge (XOR11);

                \node (k10) [above=25pt of XOR13,marked on=<5>] {\small $\mathrm{TK}(1)$};
                \path[line,marked on=<5>] (k10) edge (XOR13);

                \node (k20) [above=25pt of XOR15,marked on=<6>] {\small $\mathrm{TK}(2)$};
                \path[line,marked on=<6>] (k20) edge (XOR15);


                \node (k01) [above=25pt of XOR22,marked on=<4>] {\small $\mathrm{TK}(0)$};
                \path[line,marked on=<4>] (k01) edge (XOR22);

                \node (k11) [above=25pt of XOR24,marked on=<5>] {\small $\mathrm{TK}(1)$};
                \path[line,marked on=<5>] (k11) edge (XOR24);


                \node (k22) [above=25pt of XOR32,marked on=<6>] {\small $\mathrm{TK}(2)$};
                \path[line,marked on=<6>] (k22) edge (XOR32);

                \node (k02) [above=25pt of XOR34,marked on=<4>] {\small $\mathrm{TK}(0)$};
                \path[line,marked on=<4>] (k02) edge (XOR34);


                \node (w00) [below=25pt of XOR12,marked on=<7>] {\small $W(0)$};
                \path[line,marked on=<7>] (w00) edge (XOR12);

                \node (w10) [below=25pt of XOR13,marked on=<5>] {\small $W(1)$};
                \path[line,marked on=<5>] (w10) edge (XOR13);

                \node (w20) [below=25pt of XOR14,marked on=<7>] {\small $W(2)$};
                \path[line,marked on=<7>] (w20) edge (XOR14);

                \node (w30) [below=25pt of XOR15,marked on=<6>] {\small $W(3)$};
                \path[line,marked on=<6>] (w30) edge (XOR15);


                \node (w01) [below=25pt of XOR21,marked on=<7>] {\small $W(4)$};
                \path[line,marked on=<7>] (w01) edge (XOR21);

                \node (w11) [below=25pt of XOR22,marked on=<4>] {\small $W(5)$};
                \path[line,marked on=<4>] (w11) edge (XOR22);

                \node (w21) [below=25pt of XOR23,marked on=<7>] {\small $W(6)$};
                \path[line,marked on=<7>] (w21) edge (XOR23);

                \node (w31) [below=25pt of XOR24,marked on=<5>] {\small $W(7)$};
                \path[line,marked on=<5>] (w31) edge (XOR24);


                \node (w02) [below=25pt of XOR31,marked on=<7>] {\small $W(8)$};
                \path[line,marked on=<7>] (w02) edge (XOR31);

                \node (w12) [below=25pt of XOR32,marked on=<6>] {\small $W(9)$};
                \path[line,marked on=<6>] (w12) edge (XOR32);

                \node (w22) [below=25pt of XOR33,marked on=<7>] {\small $W(10)$};
                \path[line,marked on=<7>] (w22) edge (XOR33);

                \node (w32) [below=25pt of XOR34,marked on=<4>] {\small $W(11)$};
                \path[line,marked on=<4>] (w32) edge (XOR34);
            \end{scope}
            \end{tikzpicture}
        \end{column}
        \begin{column}{0.45\textwidth}
            \vspace*{-10pt}
            \begin{align*}
                             D &= {\only<4>{\color{alertred}}D_{\mathrm{TK}(0)}} \cup {\only<5>{\color{alertred}}D_{\mathrm{TK}(1)}} \cup {\only<6>{\color{alertred}}D_{\mathrm{TK}(2)}} \cup {\only<7>{\color{alertred}}D_0}\\[5pt]
                \visible<4->{\only<4>{\color{alertred}}D_{\mathrm{TK}(0)} &\only<4>{\color{alertred}}= \set{0 + W(5), 0 + W(11), W(5) + W(11)}} \\
                \visible<5->{\only<5>{\color{alertred}}D_{\mathrm{TK}(1)} &\only<5>{\color{alertred}}= \set{W(1) + W(7)}} \\
                \visible<6->{\only<6>{\color{alertred}}D_{\mathrm{TK}(2)} &\only<6>{\color{alertred}}= \set{W(3) + W(9)}} \\
                \visible<7->{\only<7>{\color{alertred}}D_{0} &\only<7>{\color{alertred}}= \set{a + b \given a, b \in D^\prime, a \neq b}} \\
                \visible<7->{\only<7>{\color{alertred}}D^\prime &\only<7>{\color{alertred}}= \set{W(0), W(2), W(4), W(6), W(8), W(10)}}
            \end{align*}
        \end{column}
        }
    \end{columns}
\end{frame}

\begin{frame}{Invariant Attacks}{Application to \clyde/}
    \begin{itemize}
        \item Computing $W_L$ is efficiently doable (takes $\approx$ 10 seconds on my laptop).
        \item For the round constants chosen for \clyde/, $\dim W_L(D) = 128 = n$.
    \end{itemize}
    \begin{itemize}
        \item Thus, we can apply:
              \begin{block}{Proposition~2 \cite{C:BCLR17}}
                  Suppose that the dimension of $W_L(D)$ is $n$.
                  Then any invariant $g$ is constant (and thus trivial).
              \end{block}
    \end{itemize}
    \begin{itemize}
        \item We conclude that we cannot find any non-trivial $g$ for \clyde/ which is at the same time invariant for the S-box layer and for the linear layer.
    \end{itemize}
\end{frame}

\begin{frame}{Invariant Attacks}{Improvable?}
    \begin{block}{Bounding the dimension of $W_L$,~\cite[Theorem~1]{C:BCLR17}}
        Given a linear layer $L$.
        Denote by $Q_i$ its \emph{invariant factors}.
        Then
        \begin{equation*}
            \max_{c_1, \dots, c_t \in \F_2^n} \dim W_L(c_1, \dots, c_t) = \sum_{i=1}^t \deg Q_i\;.
        \end{equation*}
    \end{block}
    \pause
    \begin{block}{Application to \clyde/}
    \begin{columns}
        \begin{column}{0.6\textwidth}
        \begin{itemize}
            \item Compute invariant factors of linear layer:
            \item This gives a lower bound on the number of rounds:
        \end{itemize}
        \end{column}
        \pause
        \begin{column}{0.35\textwidth}
        \begin{itemize}
            \item[] $4 \times (x^{32}+1)$
            \item[] 3 steps/6 rounds
        \end{itemize}
        \end{column}
    \end{columns}
    \pause
    \begin{columns}
        \begin{column}{0.475\textwidth}
            \begin{itemize}
                \item 3 stps/6 rnds: $\dim W_L(c_1,\dots,c_4) = \hphantom{1}96$
                \item 4 stps/8 rnds: $\dim W_L(c_1,\dots,c_8) = 128$
            \end{itemize}
        \end{column}
        \begin{column}{0.475\textwidth}
            \begin{itemize}
                \item 5 stps/10 rnds: $\dim W_L(c_1,\dots,c_{13}) = 128$
                \item 6 stps/12 rnds: $\dim W_L(c_1,\dots,c_{20}) = 128$
            \end{itemize}
        \end{column}
    \end{columns}
    \end{block}
\end{frame}

\section{Subspace Trails}
\begin{frame}
    \centering
    \Huge
    Subspace Trails

    {\Large
        Probability~1~Truncated Differentials
    }

    \vfill
\end{frame}
\begin{frame}{Subspace Trails}
    \begin{block}{Main Idea: Subspace Trails}
        \centering
        \vspace{0.25em}
        \begin{tikzpicture}[scale=0.8]
            \tikzstyle{every node}=[transform shape];

            \node (left2-space) [draw,rectangle,thick,rounded corners,minimum width=3cm,minimum height=4cm,fill=white] at (1,0) {};
            \draw[thick] (left2-space.west)+(0,1.125cm) -- node[above, yshift=1.5mm] {$\coset{U}{a_r}$} +(3cm,1.125cm);
            \draw[thick] (left2-space.west)+(0,-0.25cm) -- node[above, yshift=5mm] {\dots} +(3cm,-0.25cm);
            \draw[thick] (left2-space.west)+(0,-1.125cm) -- node[above, yshift=1.5mm] {$\coset{U}{a_1}$}
                                                            node[below, yshift=-1.5mm] {$U$} +(3cm,-1.125cm);

            \node (middle-space) [draw,rectangle,thick,rounded corners,minimum width=3cm,minimum height=4cm,fill=white] at (5,0) {};
            \draw[thick] (middle-space.north)+(0pt,-0.5pt) -- node[above, yshift=0.5mm, rotate=45] (vbs) {$\coset{V}{b_s}$} +(-1.5cm,-1.5cm);
            \draw[thick] (middle-space.east)+(-0.5pt,1cm) -- node[above, yshift=10mm, rotate=45] (middle-dots) {\dots} +(-3cm+1.25pt,-2cm+1.25pt);
            \draw[thick] (middle-space.east)+(-0.5pt,-5mm) -- node[above, yshift=2.5mm, rotate=45] (vb1) {$\coset{V}{b_1}$}
                                                        node[below, yshift=-0.5mm, rotate=45] (v) {$V$} +(-1.5cm,-2cm);

            \draw[-latex] (1.75,-1.5) to [bend left] (4.75,-1.5);
            \visible<2->{%
            \draw[-latex] (1.75,-0.5) to [bend left] (6.325,-1.0);
            \draw[-latex] (1.75,+1.5) to [bend left] (middle-dots);
            }

            \node (left-f) at (3,0) {$F$};

            \visible<3->{%
            \node (right-space) [draw,rectangle,thick,rounded corners,minimum width=3cm,minimum height=4cm,fill=white] at (9,0) {};
            \draw[thick] (right-space.north)+(0pt,-0.5pt) -- node[above, yshift=0.5mm, rotate=-45] (wct) {$\coset{W}{c_t}$} +(+1.5cm,-1.5cm);
            \draw[thick] (right-space.west)+(0.5pt,1cm) -- node[above, yshift=10mm, rotate=-45] (right-dots) {\dots} +(3cm-1.25pt,-2cm+1.25pt);
            \draw[thick] (right-space.west)+(0.5pt,-5mm) -- node[above, yshift=2.5mm, rotate=-45] (wc1) {$\coset{W}{c_1}$}
                                                        node[below, yshift=-0.5mm, rotate=-45] (w) {$W$} +(+1.5cm,-2cm);

            \node (middle-f) at (7,0) {$F$};
            \node (right-f) at (11,0) {$\ldots$};

            \draw[-latex] (vb1) to [bend left] (w);
            \draw[-latex] (v) to [bend left] (wc1);
            \draw[-latex] (vbs) to [bend right] (9.625,+1.75);
            }
        \end{tikzpicture}
    \end{block}
    \visible<3->{%
    \begin{block}{Subspace Trail Cryptanalysis~\cite{ToSC:GraRecRon16} (FSE'16)}
        \vspace{0.25em}
        \centering
        Let $U_0, \ldots, U_r \subseteq \F_2^n$, and $F : \F_2^n \to \F_2^n$.
        Then these form a \emph{subspace trail} (ST), $U_0 \through{F} \cdots \through{F} U_r$, iff
        \begin{equation*}
            \forall a \in U_i^\perp : \exists b \in U_{i+1}^\perp : \qquad F(\coset{U_i}{a}) \subseteq \coset{U_{i+1}}{b}
        \end{equation*}
    \end{block}
    }
\end{frame}

\begin{frame}{Computing Subspace Trails}
    \centering
    \begin{minipage}{0.45\textwidth}
    Given a starting subspace $U$, we can efficiently compute the corresponding longest subspace trail.

    \begin{lemma}
        \vspace{14pt}
        Let $U \through{F} V$ be a ST\@.
        Then for all $u \in U$ and all~$x$: $F(x) + F(x + u) \in V$.
        \vspace{14pt}
    \end{lemma}
    \end{minipage}
    \begin{minipage}{0.45\textwidth}
    \visible<2->{%
    \begin{block}{Proof}
        \centering
        \begin{tikzpicture}[scale=0.75]
            \tikzstyle{every node}=[transform shape];

            \node (left2-space) [draw,rectangle,thick,rounded corners,minimum width=3cm,minimum height=4cm,fill=white] at (1,0) {};
            \draw[thick] (left2-space.west)+(0,1.125cm) -- node[above, yshift=1.5mm] {$\coset{U}{a_s}$} +(3cm,1.125cm);
            \draw[thick] (left2-space.west)+(0,-0.25cm) -- node[above, yshift=5mm] {\dots} +(3cm,-0.25cm);
            \draw[thick] (left2-space.west)+(0,-1.125cm) -- node[below, yshift=-1.5mm] {$U$} +(3cm,-1.125cm);

            \node (right2-space) [draw,rectangle,thick,rounded corners,minimum width=3cm,minimum height=4cm,fill=white] at (5,0) {};
            \draw[thick] (right2-space.north)+(0pt,-0.5pt) -- node[above, yshift=0.5mm, rotate=45] {$\coset{V}{b_t}$} +(-1.5cm,-1.5cm);
            \draw[thick] (right2-space.east)+(-0.5pt,1cm) -- node[above, yshift=10mm, rotate=45] {\dots} +(-3cm+1.25pt,-2cm+1.25pt);
            \draw[thick] (right2-space.east)+(-0.5pt,-5mm) -- node[below, yshift=-0.5mm, rotate=45] {$V$} +(-1.5cm,-2cm);

            \node[xshift=-20pt, yshift=-18pt] (x1) at (left2-space) {$\cdot$};
            \node[above] (x1-label) at (x1) {$x$};

            \node[xshift=32pt, yshift=-4pt] (y1) at (right2-space) {$\cdot$};
            \node[above] (y1-label) at (y1) {$F(x)$};

            \draw[-latex] (x1) to [bend left] node[below,yshift=-5pt] {$F$} (y1);

            \visible<3->{%
                \node[xshift=25pt, yshift=-24pt] (x2) at (left2-space) {$\cdot$};
                \node[above] (x2-label) at (x2) {$x+u$};

                \node[xshift=-12pt, yshift=-42pt] (y2) at (right2-space) {$\cdot$};
                \node[below] (y2-label) at (y2) {$F(x+u)$};

                \draw[-latex] (x1) -- node[below,draw,fill=white,inner sep=1pt,circle] (u-label) {$u$} (x2);
                \node[xshift=17pt, yshift=-23pt] at (u-label) (u) {$\cdot$};

                \draw[-latex] (u-label) -- (u);

                \draw[-latex] (x2) to [bend left] (y2);
            }

            \visible<4->{%
                \draw[-latex] (y1) -- node[below,xshift=2pt,draw,fill=white,inner sep=1pt,circle] (v-label) {$v$} (y2);
                \node[xshift=22pt, yshift=-9pt] at (v-label) (v) {$\cdot$};

                \draw[-latex] (v-label) -- (v);
            }

        \end{tikzpicture}
    \end{block}
    }
    \end{minipage}
    \visible<5->{%
    \begin{minipage}{0.925\textwidth}
    \begin{block}{Computing the subspace trail}
        \begin{itemize}
            \item To compute the next subspace, we have to compute the image of the derivatives.
        \end{itemize}
    \end{block}
    \end{minipage}
    }
\end{frame}

\begin{frame}{Computing Subspace Trails}{Algorithm}
    \centering
    \begin{minipage}{0.7\textwidth}
    \centering
    \begin{block}{Compute Subspace Trails}
    \begin{algorithmic}[1]
        \Require{A nonlinear, bijective function $F : \F_2^n \to \F_2^n$ and a subspace $U$.}
        \Ensure{The longest ST starting in $U$ over $F$.}
        \Statex{}
        \Function{Compute Trail}{$F$, $U$}
        \If{$\dim(U) = n$}
            \State{}\Return{$U$}
        \EndIf{}
        \State{}$V \leftarrow \emptyset$
        \For{$u_i$ basis vectors of $U$}
            \For{enough $x \in_\mathrm{R} \F_2^n$}\label{alg:compute_trail_1line}
                \Comment \eg/ $n+20$ $x$'s are enough
                \State{} $V \leftarrow V \cup \Delta_{u_i}(F)(x)$\label{alg:compute_trail_2line}
                \Comment $\Delta_a(F)(x) \coloneqq F(x) + F(x+a)$
            \EndFor{}
        \EndFor{}
        \State{}$V \leftarrow \mathrm{span}(V)$
        \State{}\Return{the subspace trail $U \rightarrow \textsc{Compute Trail}(F, V)$}
        \EndFunction{}
    \end{algorithmic}
    \end{block}
    \end{minipage}
\end{frame}

\begin{frame}{Subspace Trails}{Proving Resistance}
    \begin{block}{\textbf{Goal}: Apply security argument from}
    \begin{quote}
        \fullcite{ToSC:LeaTezWie18}.
    \end{quote}
    \end{block}
    \begin{exampleblock}{What do we get from this?}
        \begin{itemize}
            \item (Tight) upper bound on the length of any ST for an SPN construction
        \end{itemize}
    \end{exampleblock}
    \begin{alertblock}{Why is the \textsc{Compute Trail} algorithm not enough?}
        \begin{itemize}
            \item Exhaustively checking all possible starting points is to costly.
        \end{itemize}
    \end{alertblock}
\end{frame}

\begin{frame}{Subspace Trails}{How to bound the length of any subspace trail}
    \begin{columns}[t,onlytextwidth]
        \begin{column}{0.35\textwidth}
            \begin{block}{Observation\vphantom{g}}
                \vspace{0.5em}
                \centering
                \begin{tikzpicture}
                \foreach \z in {0,1,2,3} {%
                    \node[] (i\z) at ($(-0.25, \z*3em)+(0,0.35)$) {};
                    \node[] (o\z) at ($(0, \z*3em)+(3em,0.35)$) {};
                    \draw[thick,solid] (i\z) -- (o\z.center);
                }

                %% SBoxes
                \foreach \z in {0,1,2,3} {%
                    \node[draw,thick,solid,minimum width=2em,minimum height=2.75em,fill=white] (sl\z) at ($(i\z) + (2em,0em)$) {$S$};
                }

                \draw [draw=none] (i1) -- node[left,xshift=-0.5em] (U) {$U$} (i2);

                \draw [draw=none] (o1) -- node[right,xshift=0.5em] (V) {$V \ \ni$} (o2);
                \node [xshift=1em] at (V.east) {$\begin{pmatrix}0 \\ \\ \alpha \\ \\ 0 \\ \\ 0 \end{pmatrix}$} (o2);
                \end{tikzpicture}
                \vspace{0.5em}
            \end{block}
        \end{column}
        \begin{column}{0.6\textwidth}
            \visible<2->{%
            \begin{block}{Algorithm Idea}
                %\begin{minipage}[t][117.5pt][t]{\textwidth}
                \vspace{0.25em}
                Compute the subspace trails for any starting point $W_{i,\alpha} \in \mathcal{W}$, with
                    \begin{equation*}
                        W_{i,\alpha} \coloneqq (\underbrace{0, \ldots, 0}_{i-1}, \alpha, 0, \ldots, 0)
                    \end{equation*}
                %\end{minipage}
            \end{block}
            \begin{block}{Complexity (Size of $\mathcal{W}$)}
                \centering
                For an S-box layer $S : \F_2^{kn} \to \F_2^{kn}$ with $k$ S-boxes, each $n$-bit:\\
                $\abs{\mathcal{W}} = k \cdot (2^n-1)$
            \end{block}
            }
        \end{column}
    \end{columns}
\end{frame}

\begin{frame}{Subspace Trails}{Algorithm}
    \centering
    \begin{block}{Generic Subspace Trail Search}
    \begin{algorithmic}[1]
        \Require{A linear layer matrix $M : \F_2^{n \cdot k \times n \cdot k}$, and an S-box $S : \F_2^n \to \F_2^n$.}
        \Ensure{A bound on the length of all STs over $F = M \circ S^k$.}
        \Statex{}
        \Function{Generic Subspace Trail Length}{$M$, $S$}
        \State{}empty list $L$
        \For{possible initial subspaces represented by $W_{i,\alpha} \in \mathcal{W}$}
            \Comment Overall $k \cdot (2^n-1)$ iterations
            \State{}$L.\mathrm{append}(\textsc{Compute Trail}(S^k \circ M, \set{W_{i,\alpha}}))$
            \Comment $S^k$ denotes the S-box layer
        \EndFor{}
        \State{}\Return{$\max{\set{\mathrm{len}(t) \given t \in L}}$}
        \EndFunction{}
    \end{algorithmic}
    \end{block}

    \begin{block}{Overall Complexity}
        \centering
        \begin{tabular}{lccccc}
            \toprule
            Algorithm  & \textsc{Compute Trail} & \textsc{Generic Subspace Trail Length} &           Overall          &  \clyde/ & \shadow/ \\
            Complexity & $\mathcal{O}(k^2 n^2)$ &          $\mathcal{O}(k2^n)$           & $\mathcal{O}(k^3 n^2 2^n)$ & $2^{23}$ & $2^{29}$ \\
            \bottomrule
        \end{tabular}
    \end{block}
\end{frame}

%\subsection{Result}
%
%\begin{table}
%    \centering
%    \caption{%
%        Bound on the longest subspace trails in \clyde/ and \shadow/ (actual longest subspace trails can be one round longer).
%    }\label{tab:results_st}
%    \renewcommand{\arraystretch}{1.2}
%    \begin{tabular}{lcccc}
%        \toprule
%                 & \multicolumn{4}{c}{\cref{alg:generic}} \\
%        Cipher   & $r_e$   &   $d$   &    $r_d$   &  $d$  \\
%        \midrule
%        \clyde/  &   2     &    44   &      2     &   41  \\ \rowcolor{gray!10}
%        \shadow/ &  ???    &   ???   &     ---    &  ---  \\
%        \bottomrule
%    \end{tabular}
%\end{table}
%
%See \cref{tab:results_st}.

\section{Division Property}
\begin{frame}
    \centering
    \Huge
    Division Property
    \vfill
\end{frame}
\begin{frame}{Division Property}
    \begin{block}{\textbf{Goal}: Apply security argument from}
    \begin{quote}
        \fullcite{AC:XZBL16}.
    \end{quote}
    \end{block}
    \begin{exampleblock}{What do we get from this?}
        \centering
        bla
    \end{exampleblock}
    \begin{block}{Approach}
        \centering
        Model division trail propagations as MILP, find solutions for this over increasing number of rounds.
    \end{block}
\end{frame}

\section{Results}
\begin{frame}
    \centering
    \Huge
    Results
    \vfill
\end{frame}
\begin{frame}{Results}{Thanks for your attention!}
    \centering
    \begin{minipage}{0.5\textwidth}
    \begin{block}{Number of rounds}
    \centering
    \renewcommand{\arraystretch}{1.2}
    \begin{tabular}{ccc}
        \toprule
        Technique         & \clyde/ & \shadow/ \\
        \midrule
        Invariants        &   6     &   ---    \\ \rowcolor{saphierblau!20}
        Subspace Trails   & 2 (+1)  &  4 (+1)  \\
        Division Property &   8     &   ---    \\
        \bottomrule
    \end{tabular}
    \end{block}
    \begin{block}{Future Work/Cryptanalysis}
        \begin{itemize}
            \item Cryptagraph~\cite{ToSC:HalVej18}
            \item Post cryptanalysis results on mailinglist?
            \item Eprint Write-Up?
        \end{itemize}
    \end{block}
    \end{minipage}
    \begin{minipage}{0.45\textwidth}
        \centering
        \begin{figure}[!htb]
            \includegraphics[height=50mm]{data/flickr/questionmark.png}
        \end{figure}
    \end{minipage}
\end{frame}

\begin{frame}[allowframebreaks]{References}
    \tiny
    \printbibliography{}
\end{frame}


\end{document}
