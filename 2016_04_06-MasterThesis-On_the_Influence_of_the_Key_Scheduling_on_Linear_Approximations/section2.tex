\section{Introduction}
\begin{frame}{\spresent/}{Introduction}
	\begin{columns}
		\begin{column}{0.48\textwidth}
			\begin{block}{\spresent/-$\bracket*{4}$}
				\centering
				\vspace{1em}
				\includegraphics[width=\maxwidth{\textwidth},
				height=\maxheight{.55\textheight},
				keepaspectratio]%
				{data/plots/smallpresent4.pdf}
				\vspace{1em}
			\end{block}
		\end{column}
		\begin{column}{0.5\textwidth}
			\begin{itemize}
				\item SPN
				\item \present/'s $4$~bit \subbox/ %choose an \subbox/
				\item Blocksize is $4 \cdot n$
				\item last round omits permutation
			\end{itemize}
			\vspace{1em}
			\begin{itemize}
				\item standard \present/: $n = 16$
			\end{itemize}
		\end{column}
	\end{columns}
\end{frame}

\begin{frame}{4~bit \subboxes/}{Introduction}
	\begin{block}{Representatives of Serpent-type Equivalence Classes}
		\centering
		\vspace{1em}
		\setlength{\tabcolsep}{4pt}
		\begin{tabular}{ccccccccccccccccc}
			\toprule
			$x$      & 0& 1& 2& 3& 4& 5& 6& 7& 8& 9&10&11&12&13&14&15\\\midrule
			$R_0(x)$ & 0& 3& 5& 6& 7&10&11&12&13& 4&14& 9& 8& 1& 2&15\\
			$R_1(x)$ & 0& 3& 5& 8& 6& 9&10& 7&11&12&14& 2& 1&15&13& 4\\
			$R_2(x)$ & 0& 3& 5& 8& 6& 9&11& 2&13& 4&14& 1&10&15& 7&12\\
			$\vdots$ & & & & & & & & & $\cdots$ & & & & & & & $\vdots$\\
			\bottomrule
		\end{tabular}
		\vspace{1em}
	\end{block}
	\begin{itemize}
		\item all $4$~bit \subboxes/ are classified
		\item 16 \emph{optimal} and 20 \emph{Serpent-type} equivalence classes
	\end{itemize}
\end{frame}

\begin{frame}{Linear Cryptanalysis (LC)}{Introduction}
	\begin{columns}[T]
		\begin{column}{.58\textwidth}
			\begin{itemize}
				\item invented by Matsui 1993--1994
				\item broke DES
				\item together with Differential Cryptanalysis (DC) most used attack on block ciphers
			\end{itemize}
			\vspace{1em}
			\begin{itemize}
				\item advanced techniques: multidimensional LC, zero-correlation LC, \ldots
				\item links to DC%~\cite{blondeau_nyberg/links}
			\end{itemize}
		\end{column}
		\hfill
		\begin{column}{.38\textwidth}
			\begin{figure}[!ht]
				\includegraphics[height=50mm]{data/matsui.jpg}
			\end{figure}
		\end{column}
	\end{columns}\blfootnote{\scriptsize Image: \url{http://www.isce2009.ryukoku.ac.jp/eng/keynote_address.html}}
\end{frame}

\begin{frame}{Linear Approximations}{Introduction}
	\vspace{-10pt}
	\visible<1->{%
	\begin{itemize}
		\item We want to linear approximate a function $F : \mathbb{F}_2^n \rightarrow \mathbb{F}_2^n$
	\end{itemize}
	}

	\vspace{-10pt}
	\begin{columns}[T]
	\visible<2->{%
		\begin{column}{0.48\textwidth}
			\begin{block}{Dot-Product}
				\begin{equation*}
				\vspace{-2pt}
					\langle \alpha, x \rangle = \bigoplus_{i=0}^{n-1} \alpha_i x_i
				\end{equation*}
			\end{block}
		\end{column}
	}
	\visible<3->{%
		\begin{column}{0.48\textwidth}
			\begin{block}{Mask}
				Let $\alpha, \beta, x \in \mathbb{F}_2^n$ and
				%\vspace{-6pt}
				\begin{equation}
					\langle \alpha, x \rangle = \langle \beta, F(x) \rangle \label{equ:masks}
				\end{equation}
			\end{block}
		\end{column}
	}
	\end{columns}

	\visible<3->{%
	\begin{itemize}
		\item We say $\alpha$ is an \emph{input mask} and $\beta$ is an \emph{output mask}.
		\item Equation~\ref{equ:masks} does not hold for every input/output masks.
	\end{itemize}
	}
	\vspace{1em}
	\visible<4->{%
	\begin{itemize}
		\item It is \emph{correlated}, i.e., $\text{Pr}[\langle \alpha, x \rangle = \langle
			\beta, F(x) \rangle] = \frac{c(\alpha, \beta) - 1}{2}$.
	\end{itemize}
	}
\end{frame}

\begin{frame}{LC Example: \spresent/}{Introduction}
	\only<-4>{%
		\begin{columns}[T]
			\begin{column}{0.50\textwidth}
			\only<-2>{%
				\begin{block}{\spresent/-$\bracket*{4}$ over 3 Rounds}
					\centering
					\includegraphics[width=\maxwidth{.75\textwidth},
									height=\maxheight{.75\textheight},
									keepaspectratio]%
									{data/plots/smallpresent4_3r.pdf}
				\end{block}
			}
			\only<3-4>{%
				\begin{block}{\spresent/-$\bracket*{4}$ over 3 Rounds}
					\centering
					\includegraphics[width=\maxwidth{.75\textwidth},
									height=\maxheight{.75\textheight},
									keepaspectratio]%
									{data/plots/smallpresent4_3r_hl.pdf}
				\end{block}
			}
			\end{column}
			\begin{column}{0.48\textwidth}
			\vspace{1cm}
			\visible<2-4>{%
				Basically approximate:
				\begin{itemize}
						\item the \subbox/
						\item the linear layer
				\end{itemize}
			}
			\vspace{1em}
			\visible<4>{%
				\begin{itemize}
						\item the linear layer \enquote{is easy}
						\item for the \subboxes/ use \emph{Linear Approximation Table (LAT)}
				\end{itemize}
			}
			\end{column}
		\end{columns}
	}

	\only<5->{%
		\begin{block}{LAT}
			\robustify\diagbox
			\robustify\hphantom
			\centering
			\scriptsize
			\vspace{1em}
			\tabcolsep=-2pt
			\begin{tabular}{SSSSSSSSSSSSSSSSS}
				\toprule
				\diagbox{$\beta$\hphantom{.}}{\hphantom{.}$\alpha$}&
				    1& 2& 3& 4& 5& 6& 7& 8& 9&10&11&12&13&14&15 \\
				1 &  &  &  &  &-8&  &-8&  &  &  &  &  &-8&  & 8 \\
				2 &  & 4& 4&-4&-4&  &  & 4&-4&  & 8&  & 8&-4&-4 \\
				3 &  & 4& 4& 4&-4&-8&  &-4& 4&-8&  &  &  &-4& 4 \\
				4 &  &-4& 4&-4&-4&  & 8&-4&-4&  &-8&  &  &-4&   \\
				5 &  &-4& 4&-4& 4&  &  & 4& 4&-8&  & 8&  & 4&   \\
				6 &  &  &-8&  &  &-8&  &  &-8&  &  & 8&  &  &-4 \\
				7 &  &  & 8& 8&  &  &  &  &-8&  &  &  &  & 8&-4 \\
				8 &  & 4&-4&  &  &-4& 4&-4& 4&  &  &-4& 4& 8&-4 \\
				9 & 8&-4&-4&  &  & 4&-4&-4&-4&-8&  &-4& 4&  & 4 \\
				10&  & 8&  & 4& 4& 4&-4&  &  &  &-8& 4& 4&-4& 8 \\
				11&-8&  &  &-4&-4& 4&-4&-8&  &  &  & 4& 4& 4&   \\
				12&  &  &  &-4&-4&-4&-4& 8&  &  &-8&-4& 4& 4& 4 \\
				13& 8& 8&  &-4&-4& 4& 4&  &  &  &  & 4&-4& 4& 4 \\
				14&  & 4& 4&-8& 8&-4&-4&-4&-4&  &  &-4&-4&  &   \\
				15& 8& 4&-4& 4& 4&  &  & 8&  & 4&-4&-4&-4&  &   \\
				\bottomrule
			\end{tabular}
			\vspace{1em}
		\end{block}
	}
\end{frame}

\begin{frame}{Linear Hull}{Introduction}
	\visible<1->{%
		\begin{itemize}
			\item Our example exhibits more than one trail for $(\alpha, \beta) = (15, 15)$
			\item Key dependency
		\end{itemize}
	}
	\visible<2->{%
		\begin{block}{Linear Hull}
			Let $F : \mathbb{F}_2^n \rightarrow \mathbb{F}_2^n$ be a block cipher over $r$~rounds,\newline
			and $E : \mathbb{F}_2^m \rightarrow {\left(\mathbb{F}_2^n\right)}^{r+1}$ a key schedule.
			The \emph{linear hull} $c_F^k(\alpha,\beta)$ is
			\begin{equation*}
				c_F^k(\alpha,\beta) := \sum_{\theta|\theta_0=\alpha,\theta_r=\beta} {(-1)}^{\langle \theta, E(k) \rangle} c_\theta
			\end{equation*}
		\end{block}
	}
\end{frame}

\begin{frame}{Distributions}{Introduction}
	\visible<1->{%
	\begin{itemize}
		\item Attack complexity of linear cryptanalysis is proportional to $\left(c_\theta\right)^{-2}$.
	\end{itemize}
	}
	\vspace{1em}
	\visible<2->{%
	\begin{itemize}
		\item We assume the \emph{Hypothesis of Stochastic equivalence}.
		\item Thus, distribution of linear biases follows a normal distribution.
		\item Its width is defined by the variance.
	\end{itemize}
	}
	\vspace{1em}
	\visible<3->{%
	\begin{itemize}
		\item What happens with different key schedules?
	\end{itemize}
	}
\end{frame}
